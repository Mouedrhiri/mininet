\documentclass{beamer}
\usepackage[utf8]{inputenc}
\usepackage[french]{babel}
\usepackage{graphicx}
\usetheme{Madrid}
\title{MiniNet  Présentation C\&C}
\author{Équipe Copilot}
\date{Décembre 2025}
\begin{document}
\frame{\titlepage}
\begin{frame}{Objectifs}
\begin{itemize}
\item Prototype pédagogique de mini réseau social
\item Illustrer l'architecture Composants \& Connecteurs
\item Traçabilité ACME/PlantUML \textrightarrow{} code
\end{itemize}
\end{frame}
\begin{frame}{Exigences}
\begin{enumerate}
\item Inscription / authentification
\item Posts (texte, image URL, lien)
\item Fil des amis
\item Messages privés
\item Gestion des amis
\end{enumerate}
\end{frame}
\begin{frame}{Composants}
APIFrontend, UserManager, PostManager, MessageService, NotificationService, Storage
\end{frame}
\begin{frame}{Connecteurs}
REST (Express), EventBus (in-memory)
\end{frame}
\begin{frame}{ACME}
backend/src/acme/mininet.acme.json
\end{frame}
\begin{frame}{Diagramme PlantUML}
\includegraphics[width=\textwidth]{images/demo1.png}
\end{frame}
\begin{frame}{Démo}
\texttt{bash backend/scripts/demo.sh}
\end{frame}
\begin{frame}{Choix techniques}
Express, EventBus, SQLite
\end{frame}
\begin{frame}{Évolutions possibles}
Auth sécurisée, files externes, tests, monitoring
\end{frame}
\begin{frame}{Conclusion}
Prototype exécutable et traçabilité claire.
\end{frame}
\end{document}
