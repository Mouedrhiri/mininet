\documentclass[12pt]{article}
\usepackage[utf8]{inputenc}
\usepackage[french]{babel}
\usepackage{graphicx}
\usepackage{hyperref}
\usepackage{geometry}
\usepackage{longtable}
\geometry{margin=2.5cm}
\title{MiniNet  Prototype pédagogique C\&C}
\author{Équipe Copilot}
\date{Décembre 2025}
\begin{document}
\maketitle
\section{Introduction}
MiniNet est un prototype minimal de réseau social implémenté pour illustrer une architecture Composants \& Connecteurs (C\&C). Le but principal est de démontrer la traçabilité entre la spécification (ACME/PlantUML) et le code, avec une pile technique facile à exécuter (Node.js, Express, SQLite, EventBus in-memory).
\section{Exigences}
\begin{enumerate}
  \item Inscription / authentification simple.
  \item Publier des messages courts (texte, image par URL, lien).
  \item Consulter les publications des amis (fil).
  \item Envoyer / recevoir messages privés.
  \item Ajouter / supprimer amis.
\end{enumerate}
\section{Architecture Composants \& Connecteurs}
L'architecture distingue les composants fonctionnels (UserManager, PostManager, MessageService, Storage, NotificationService, APIFrontend) et les connecteurs (REST et EventBus).
\subsection{Composants}
\begin{itemize}
  \item \textbf{APIFrontend}: sert les pages statiques et expose l'API REST.
  \item \textbf{UserManager}: gestion utilisateurs et relations d'amitié.
  \item \textbf{PostManager}: création et lecture des posts.
  \item \textbf{MessageService}: messagerie privée.
  \item \textbf{NotificationService}: publication d'événements de notification.
  \item \textbf{Storage}: persistence SQLite.
\end{itemize}
\subsection{Connecteurs}
\begin{itemize}
  \item \textbf{REST}: backend/src/connectors/restConnector.js
  \item \textbf{EventBus}: backend/src/connectors/eventBus.js
\end{itemize}
\section{Spécification ACME}
Voir backend/src/acme/mininet.acme.json.
\section{Diagramme PlantUML}
Voir docs/plantuml/archi.puml.
\begin{center}
\includegraphics[width=0.8\textwidth]{images/demo1.png}
\end{center}
\section{Justification des choix}
Express/REST pour simplicité, EventBus pour notifications, SQLite pour persistence locale.
\section{Traçabilité Architecture \textrightarrow{} Code}
\begin{longtable}{p{0.28\linewidth}p{0.65\linewidth}}
\textbf{Composant/Connecteur} & \textbf{Fichier(s) Code} \\
\hline
APIFrontend & backend/src/app.js, frontend/static/* \\
UserManager & backend/src/controllers/userManager.js \\
PostManager & backend/src/controllers/postManager.js \\
MessageService & backend/src/controllers/messageService.js \\
NotificationService & backend/src/controllers/notificationService.js \\
Storage & backend/src/storage/storage.js \\
REST & backend/src/connectors/restConnector.js \\
EventBus & backend/src/connectors/eventBus.js \\
\hline
\end{longtable}
\section{Traçabilité Exigences \textrightarrow{} Fichiers}
\begin{longtable}{p{0.28\linewidth}p{0.65\linewidth}}
\textbf{Exigence} & \textbf{Implémentation (fichiers)} \\
\hline
Inscription / authentification & backend/src/controllers/userManager.js, backend/src/app.js, frontend/static/* \\
Posts (texte/image/lien) & backend/src/controllers/postManager.js, backend/src/storage/storage.js, backend/src/app.js \\
Fil des amis & backend/src/controllers/postManager.js, backend/src/app.js, frontend/static/* \\
Messages privés & backend/src/controllers/messageService.js, backend/src/app.js, frontend/static/* \\
Ajouter / supprimer amis & backend/src/controllers/userManager.js, backend/src/app.js, frontend/static/* \\
\hline
\end{longtable}
\section{Démonstration}
Le script backend/scripts/demo.sh exécute une séquence d'appels curl.
\section{Conclusion}
MiniNet montre comment structurer un prototype C\&C et assurer une traçabilité explicite.
\end{document}
